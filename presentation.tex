\documentclass{beamer}
\usetheme[alternativetitlepage=true]{Torino}

\author{Scott Moser}
\title{LXC / Linux Containers}
\institute{Michigan!/usr/group}
\date{February 11, 2014}

\begin{document}

\begin{frame}[t,plain]
    \titlepage
\end{frame}

\begin{frame}
   \frametitle{The Plan}
   \begin{itemize}
      \item What is LXC? Linux Containers?
      \item linux kernel containment features
      \item lxc demo
      \item questions
   \end{itemize}
\end{frame}

\begin{frame}
   \frametitle{Linux Containers?}
   \begin{itemize}
      \item Wikipedia: LXC (LinuX Containers) is an operating system-level virtualization method for running multiple isolated Linux systems (containers) on a single control host.
      \item linuxcontainers.org: LXC is a userspace interface for the Linux kernel containment features.  Through a powerful API and simple tools, it lets Linux users easily create and manage system or application containers.
      \item "chroot on steroids"
      \item "it's like bsd jails"
      \item from the inside looks like a vm
      \item from the outside looks like processes
    \end{itemize}
\end{frame}

\begin{frame}
   \frametitle{Linux Kernel Containment features}
   \begin{itemize}
      \item Kernel namespaces (ipc, uts, mount, pid, network and user). Absent: sys, proc
      \item Apparmor and SELinux profiles
      \item Kernel capabilities (sys\_boot, sys\_chroot, sys\_module, net\_bind\_service, sys\_mknod)
      \item Control groups / cgroups (limit memory, accounting/throttling, limit device node operations)
      \item Chroots (using pivot\_root)
      \item Seccomp policies
   \end{itemize}
\end{frame}

\begin{frame}
   \frametitle{"Hello World"}
   \begin{itemize}
      \item sudo apt-get install --quiet --assume-yes cgmanager lxc
      \item sudo lxc-create --template=cirros -n source-cirros  -- --version=devel
      \item sudo lxc-clone --snapshot source-cirros c1
      \item sudo lxc-start -n c1
      \item sudo lxc-ls --fancy
      \item ssh cirros@10.0.3.240
      \item sudo lxc-stop -n c1
      \item sudo lxc-destroy -n c1
   \end{itemize}
\end{frame}

\begin{frame}
   \frametitle{"Download Template"}
   \begin{itemize}
      \item sudo lxc-create --template=download -n x -- --list
      \item sudo lxc-create --template=download -n source-sid -- --dist=debian --release=sid --arch=amd64
      \item sudo lxc-clone --snapshot source-sid s1
      \item sudo lxc-start -n s1
   \end{itemize}
\end{frame}

\begin{frame}
   \frametitle{"We don't need no stinking sudo"}
   \begin{itemize}
      \item sudo apt-get install --assume-yes cgmanager
      \item sudo reboot
      \item sudo usermod --add-subuids 100000-165536 \$USER
      \item sudo usermod --add-subgids 100000-165536 \$USER
      \item echo "\$USER veth lxcbr1 10" | sudo tee -a /etc/lxc/lxc-usernet
   \end{itemize}
\end{frame}

\begin{frame}
   \frametitle{"OK, *now* we don't need no stinking sudo"}
   \begin{itemize}
      \item mkdir -p ~/.config/lxc
      \item cp files/user-default.conf ~/.config/lxc/default.conf
      \item lxc-create -t download -n t1 -- -d ubuntu -r trusty -a amd64
      \item lxc-start -n t1
   \end{itemize}
\end{frame}


\begin{frame}
   \frametitle{"Ubuntu Cloud"}
   \begin{itemize}
      \item sudo lxc-create -t ubuntu-cloud -n source-saucy-cloud -- --release=saucy
      \item time sudo lxc-clone -o source-saucy-cloud -n sc1 -- --userdata=files/my-userdata
      \item time sudo lxc-clone --snapshot -o source-saucy-cloud -n sc2 -- --userdata=files/my-userdata
      \item sudo lxc-start -n sc1
   \end{itemize}
\end{frame}

\begin{frame}
   \frametitle{Resources}
   \begin{itemize}
      \item \href{http://linuxcontainers.org}{linuxcontainers.org}
      \item \href{https://www.stgraber.org/2013/12/20/lxc-1-0-blog-post-series/}{10 part series on LXC 1.0}
      \item \href{http://lwn.net/Articles/515034/}{LWN.net Secure Linux containers}
   \end{itemize}
\end{frame}

\end{document}
